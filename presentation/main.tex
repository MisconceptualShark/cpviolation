%% Requires compilation with XeLaTeX 
\documentclass[10pt,xcolor={table,dvipsnames},t]{beamer}
\usetheme{UCBerkeley}

\usepackage{graphics,graphicx,float}
\usepackage{multicol}
\usepackage{cancel}

\title{CP Violation In and Beyond The Standard Model}
\subtitle{Two Higgs Doublet Model Type II Corrections to Flavour Observables}
\author{Matthew Rossetter}
\institute{}
\date{\today}

\graphicspath{{../images/}}

\begin{document}

\section{Introduction}
\begin{frame}
  \titlepage
\end{frame}

% Uncomment these lines for an automatically generated outline.
%\begin{frame}{Outline}
%  \tableofcontents
%\end{frame}
\begin{frame}{The Standard Model}
    \begin{itemize}
        \item One of the great achievements of the 20th Century, the Standard Model:
            \begin{align}
                \mathcal{L} &= \underbrace{-\frac14 F_{\mu\nu}F^{\mu\nu}}_{\text{\small gauge fields}} \underbrace{+ i\bar{\Psi}\cancel{D}\Psi}_{\text{\small fermions}} \underbrace{+ (D_\mu\Phi)^\dagger(D^\mu\Phi) - V(\Phi)}_{\text{\small Higgs}} \underbrace{- Y_{ij}\bar{\Psi}_i\Phi\Psi_j + h.c.}_{\text{\small Yukawa}}
            \end{align}
        \item A gauge field theory describing matter and its interactions with 25 fundamental particles
        \item Each particle is described by a field transforming under the gauge groups of the Standard Model: SU(3)$_c\otimes$SU(2)$_L\otimes$U(1)$_Y$
        \item Has successfully described most particle phenomena we have observed to date
    \end{itemize}
\end{frame}

\begin{frame}{Unsolved Problems of the Standard Model}
    \begin{itemize}
        \item Quantum gravity; Dark matter; Neutrino masses
        \item Deviations between experiment and theory, e.g. $\mathcal{R}(K^{(*)})$
        \item Sakharov Criteria for Baryogenesis:
            \begin{enumerate}
                \item Baryon Number Violation - found in Sphalerons
                \item C and CP Violation - present but not enough
                \item First Order Phase Transition - only if $m_h<60\,$GeV 
            \end{enumerate}
    \end{itemize}
    To answer these questions, we need to consider models to extend our physics Beyond the Standard Model. 
    These models should:
    \begin{itemize}
        \item preserve predictions in agreement with experiment
        \item agree with experimental bounds
        \item follow the structures of gauge field theory for a physical model, e.g. renormalisability
    \end{itemize}
\end{frame}

\begin{frame}{The Two Higgs Doublet Model Type II}
    \begin{columns}[t]
        \begin{column}{0.5\textwidth}
            In the Standard Model:
            \begin{itemize}
                \item One complex Higgs doublet, 4 scalar fields:
                    \begin{align}
                        \Phi_1 &= \begin{pmatrix} \phi_1 + i\phi_2 \\ \phi_0+i\phi_3\end{pmatrix}
                    \end{align}
                \item 3 fields ``eaten" by $W^\pm,Z$ bosons;\\ 1 real field left, $h$
                \item Introduce the Hermitian conjugate for masses of all fermions
            \end{itemize}
        \end{column}
        \begin{column}{0.5\textwidth}
            In 2HDM:
            \begin{itemize}
                \item Add a second doublet, now 8 scalar fields
                    \begin{align}
                        \Phi_2 &= \begin{pmatrix} \phi_5 + i\phi_6 \\ \phi_4 + i \phi_7\end{pmatrix}
                    \end{align}
                \item Now 5 fields left - $H^\pm,H^0,h^0,A^0$
                \item No need for Hermitian conjugate
                \item In Type II, $\Phi_1$ couples to down quarks; $\Phi_2$ to up quarks and charged leptons
            \end{itemize}
        \end{column}
    \end{columns}
\end{frame}

\begin{frame}{Why Two Higgs Doublet Model?}
    \begin{itemize}
        \item Minimal Extension to SM
        \item Limited number of new parameters:
            \begin{itemize}
                \item Masses of $H^\pm,H^0,A^0$; VEV ratio $\tan\beta=\frac{v_2}{v_1}$; scalar mixing angle
            \end{itemize}
        \item Sakharov Criteria:
            \begin{enumerate}
                \item Baryon Number Violation - Sphalerons
                \item C and CP violation - more of it
                \item First Order Phase Transition - now present
            \end{enumerate}
        \item Charged weak currents gain additional decay paths, replacing $W^\pm$ with $H^\pm$ - allows for easy constraining
    \end{itemize}
    \begin{figure}[H]
        \centering
        \includegraphics[scale=0.6]{../presentation/fine1.pdf}
    \end{figure}
\end{frame}

\section{Global Fits}
\begin{frame}{First Inputs}
    \begin{columns}[T]
        \begin{column}{0.33\textwidth}
            \vspace{1.5em}
            \begin{itemize}
                \item $1\sigma$ scan
                \item Leptonic, mixing, and radiative
                \item No real constraint on $\tan\beta$
                \item $m_{H^+} > 340\,$GeV
            \end{itemize}
        \end{column}
        \begin{column}{0.60\textwidth}
            \includegraphics[scale=0.35]{global}
        \end{column}
    \end{columns}
\end{frame}
            %Initial stuff from 0907.5135 here.

\begin{frame}{New Inputs}
    \begin{columns}[c]
        \begin{column}{0.45\textwidth}
            \begin{itemize}
                \item $B_s\to\mu^+\mu^-$ 
                    \includegraphics[scale=0.25]{mumu}
            \end{itemize}
        \end{column}
        \begin{column}{0.45\textwidth}
            \begin{itemize}
                \item $R(D^{(*)})$ BOTH 
                    \includegraphics[scale=0.25]{rd_196sig}
            \end{itemize}
        \end{column}
    \end{columns}
\end{frame}

\begin{frame}{Statistical Fitting of Scans}
    \begin{columns}[T]
        \begin{column}{0.35\textwidth}
            \vspace{1.5em}
            \begin{itemize}
                \item Scanning
                \item Sigma
                \item Chi
            \end{itemize}
        \end{column}
        \begin{column}{0.6\textwidth}
            \includegraphics[scale=0.35]{global_08}
        \end{column}
    \end{columns}
\end{frame}

\begin{frame}{Extended Global Fit}
    \begin{columns}[T]
        \begin{column}{0.37\textwidth}
            \vspace{1.5em}
            \begin{itemize}
                \item 95\% CL: $m_{H^+}>390\,$GeV
                \item $1\sigma$: $m_{H^+}>530\,$GeV
                \item $\tan\beta\gtrapprox2$
            \end{itemize}
        \end{column}
        \begin{column}{0.60\textwidth}
            \includegraphics[scale=0.35]{global_lines.png}
        \end{column}
    \end{columns}
\end{frame}

\begin{frame}{CKM Element Modifications}

\end{frame}

\section{Extension to SM4}
\begin{frame}{Four Generations?}

\end{frame}

\begin{frame}{SM4 with 2HDM Type II}

\end{frame}

\section{Questions}
\begin{frame}
    \begin{center}
        \vspace{60pt}
        \Huge Any Questions?
    \end{center}
\end{frame}

\end{document}
